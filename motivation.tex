\section{Motivation}\label{s:motivation}

\begin{comment}
% MOVED --------------------------------------------------vvvvvvvvvvvvvvvv
Logical and physical fragmentation can occur independently, and appropriate measures are needed to address each issue.

% Fragmentation in device types
% HDD: logical
In the era of HDDs, the main cause of performance degradation due to file fragmentation is seek time and rotational latency.
In HDDs, there is a need for seek time to move the disk head to the track where the requested sector is located, as well as rotational latency to find it on that track.
Fragmentation increases the overhead by requiring the disk head to move more frequently, particularly having a pronounced negative impact on read operations.
Write performance can be optimized with prefetching and bufferred operations.

% SSD: logical + physical
In flash memory, there is no seek time or rotational latency, so early researchers generally argued that SSD performance is not affected by filesystem fragmentation.
Therefore, techniques like defragmentation were thought to have only negative impacts by reducing the lifespan of flash memory due to additional write operations~\cite{defrag-mobile:atc17,fragpicker:sosp21,defrag-lfs:apsys16,no-afraid:fast24,parallel-defrag:sac22,Defragmentation_read_collision}.
However, more recent studies have shown that filesystem aging and resultant fragmentation can also lead to performance degradation in SSDs~\cite{Problem_in_SSD_Empirical,senescence:fast17,Problem_in_SSD_Mobile_Devices,survey:ictc23}.
For example, in the storage devices built with NAND flash memory, data can be distributed across multiple channels, which directly impacts I/O parallelism.
built with high-density high-performance media with complicated architecture.

The first reason for this is request fragmentation due to logical fragmentation.
In the Linux I/O stack, a single I/O request can only represent contiguous LBA.
Fragmentation influences the non-contiguous storage of data, causing a single I/O request to be divided into multiple smaller I/O requests.\cite{IO}
This increases the total number of I/O requests, and all requests derived from a single I/O must be completed before the next process can proceed, leading to issues in I/O performance.
If multiple I/O requests are necessary for one file, the system must generate bio structures, request structures, and I/O commands corresponding to the number of requests.
This process incurs overhead in I/O request management.

Secondly, SSDs enhance read performance through prefetching, which loads subsequent data in advance based on LBA.
Fragmentation can cause this prefetching to load unnecessary data.\cite{defrag-lfs:apsys16} For these reasons, filesystem fragmentation is also a concern in SSDs.
% MOVED --------------------------------------------------^^^^^^^^^^^^^^^^


Despite the emergence of various new types of storage devices developed in response to the increasing demand for high-performance and efficient storage solutions, the issue of fragmentation continues to persist in storage devices.
This problem of fragmentation can lead to degraded performance and aging of the storage devices.

When new research is conducted, new SSDs are expected to have the best performance with no fragmentation whatsoever.
However, SSDs used in real life often experience performance degradation due to prolonged use and significant fragmentation.
To address this, SSDs are aged to induce the desired level of fragmentation, allowing for a more accurate representation of the conditions in which SSDs are used.

In the case of HDDs, aging is relatively straightforward because the physical area and logical area coincide.
In contrast, SSDs cannot be accessed externally due to security reasons related to the Flash Translation Layer (FTL), necessitating indirect aging.
This means that researchers cannot directly modify data placement and must perform aging by reading and writing meaningless data to the SSD for several days.
This process is time-consuming and demands a lot of efforts to incorporate the fragmentation into research.

% What we need: fast direct approach
This motivation led to the idea of reducing the time required for aging by directly aging the SSD, retrieving data from the FTL, and storing it in files.
These files can then be loaded into the FTL of a new SSD to quickly configure an aged SSD.
To achieve this, we attempted to store the FTL information and storage space data of a virtual SSD on a storage emulator called NVMeVirt, allowing for loading into a new SSD.
We argue that this approach will significantly reduce the time spent on aging tasks conducted to align with the actual user environment before research, making it an essential tool for researchers.
\end{comment}

% Fragmentation becomes more pronounced in high performance systems

% 1. Interfacing overhead is getting a larger portion
% 2. Complicated architectures make unexpected additional fragmentations

% However, studying fragmentation is difficult

% 1. Time a lot of efforts:
%    - Take weeks/months/years to age as you wants
%    - Running an experiment will distort the condition, setting again take long
%
% 2. Reproductability
%    - No standard procedure
%    - Different run-to-run results make it difficult to reproduce evaluation
%
% 3. Both logical and physical fragmentation matters
%    - Due to the hw-built nature of storage...
%

% Fragmentation is still problematic and is getting more complicated ever
It is well-known that storage systems used in real-world workloads often experience significant performance degradation due to substantial fragmentation~\cite{NEEDED}.
In modern storage systems, fragmentation poses even more challenges.
%
In the past, implications of fragmentation were relatively straightforward.
For instance, HDDs are fundamentally composed of mechanical and electromagnetic components.
Although they rely on the logical block address (LBA) scheme, which is supposed to hide the actual data placement within the device, the LBA space is tightly coupled with physical components in practice.
This leads to closely aligned layouts between logical and physical address spaces, making \emph{logical fragmentation} the most obvious but significant factor impacting device performance.
% Situation in SSDs
In contrast, SSDs operate entirely on electrical principles.
Due to the high I/O performance of flash memory, the overhead of I/O interfacing becomes increasingly significant, making the logical fragmentation still impactful.
Additionally, the highly complicated internal architectures of SSDs often result in a logical placement of data in LBA space that bears little to no correlation with its actual physical location within the device.
This decoupling introduces a greater risk of \emph{physical fragmentation}, which is becoming an increasing concern in modern storage systems~\cite{NEEDED,NEEDED}.


% For a real-world performance evaluation, researchers should consider the fragmented storage.
In this sense, we argue that fragmentation presents more nuanced performance implications in modern storage systems and requires to be considered from both the logical and physical dimensions.
However, exisiting approaches and studies lack the ability to properly incorporate the real effects of fragmentation.

% 1. Efforts in time
First, preconditioning storage systems demands a significant amount of time and effort, which is often not feasible.
Given the substantial impact of fragmentation on storage systems, it is most desirable to evaluate performance on aged systems.
Accordingly, several tools have been proposed to facilitate controlled and realistic aging~\cite{tools,studies}.
However, this process typically takes a few hours as presented in Section~\ref{s:eval}.
In a real-world system, aging may take even weeks to several months, which is critical for research efficiency.
Therefore, many studies conduct evaluation on new system conditions, failing to properly incorporate the effects of fragmentation.
This is not ideal for accurate and fair research evaluation.

% 2. Physical fragmentation
Second, prior preconditioning techniques are unable to control the physical fragmentation.
In HDDs, the closely aligned layouts between logical and physical address spaces implies small undiscovered physical fragmentation.
However, in SSDs, the complicated and parallel architectures decouple the logical and physical address spaces, increasing the likelihood of physical fragmentation.
To make matters worse, controlling the placement of data in SSDs is limited, as the internal FTL is neither accessible nor modifiable in practice.
Therefore, there is no explicit way to induce a desired degree of physical fragmentation with realistic workloads.
As an alternative, physical fragmentation is indirectly induced by blindly running operations with the hope of fragmenation or by writing data in unrealistic sequences tailored to a specific FTL.
This leaves physical fragmentation relatively underexplored in the existing literature.
Indeed, to the best of the authors' knowledge, no tool currently exists that can perform aging while controling physical fragmentation.

% 3. Adaptibility to new type of storage devices
Last, the emergence of various new storage device types has diversified the procedures required for aging storage systems.
Recently, many novel types device types have been introduced in response to the growing demand for high-performance and efficient data storage~\cite{NEEDED,NEEDED,NEEDED}.
While fragmentation continues to contribute to system aging system and degrade performance, the incresing diversity in device interface and type-specific characteristics renders traditional filesystem-level aging techniques insufficient or even inapplicable.

For instance, some devices, such as key-value SSDs and in-storage processing SSDs, communicate with the host using non-traditional interface that do not follow the standard LBA scheme.
Others, like zoned namespace SSDs and flexible data placement SSDs, require the host to issue requests under strict operation constraints.
Despite this evolving landscape, most existing aging tools are designed to operate on conventional filesystem setups only, making them ill-suited for these more complex storage environments.


% What we need: fast direct approach
These observation motivated us to explore a new approach to reduce the time required for aging SSDs.
The key idea is to extract both user data and FTL metadata from an aged SSD and then reload this snapshot into the FTL of a new SSD instance.
This process effectively replicates the entire aged state without the need for time-consuming, effort-intensive workload-based aging processes.
To realize this concept, we leverage a storage emulator called NVMeVirt.
In doing so, we can drastically reduce the time required to prepare SSDs that reflect real-world usage conditions, making this approach a valuable tool for researchers conducting performance evaluations under realistic fragmented and aged storage conditions.
